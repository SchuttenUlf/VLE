\section{Zusammenfassung und Fazit}
\label{sec:zusammenfassung}

%Die experimentellen Messdaten kommen sehr nah an die Literaturwerte heran, wie aus \ref{dia:p/Tmess} hervorgeht. Das Verhalten des Isopropanols kann sehr exakt durch die Antoine-Gleichung ausgedrückt werden. Teilweise waren die Literaturwerte zum Vergleich wenig geeignet. Das erschwert die Einschätzung. Die Verwendung der Computerprogramme ZUST und BARO ermöglichte sehr präzise und belastbare Berechnungen. Noch aussagekräftigere Ergebnisse hätten durch eine mehrfach wiederholte Messung erhalten werden können. Auch eine kleinere Schrittweite bei der Einstellung der Drücke wäre sinnvoll, um sich noch genauer an die Realität anzunähern. Von Extrapolationen über den untersuchten Bereich hinaus ist dringend abzuraten.