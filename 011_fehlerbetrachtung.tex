\section{Fehlerbetrachtung}
\label{sec:fehler}

Aus dem Brechungsindex der ersten Probe für den reinen Ethanol ergab sich durch die Kalibrierfunktion an Anteil größer von 1,017 und damit größer als eins. Das ist unmöglich. Daher wurde für die weiteren Berechnungen der theoretisch und praktisch einzig vertretbare Wert von eins angenommen. Nur so konnten sinnvolle Ergebnisse für die weiteren Berechnungen erhalten werden. Die Korrektur ist auch dahingehend gut vertretbar, dass die Abweichung nur sehr gering war.\\


Die Temperatur Hat großen einfluss auf die vorgenommenen Messungen. Schon kleine Differenzen wirken sich merklich auf die erhaltenen Ergebnisse aus. Von Bedeutung ist dabei vor allem die Temperatur bei welcher Kalibriert wurde, die am elektrischen Thermometer in der Zirkulationsapparatur abgelesene Temperatur und die Temperatur des Abbe-Refraktometers.

Ebenso wie die Temperatur wirkt sich auch die Wartezeit zwischen den Probenahmen direkt auf die Ergebnisse aus. Je länger der Zwischenraum, desto mehr Zeit hat die Anlage für eine gute Durchmischung und die Einstellung des thermischen Gleichgewichtes zu sorgen.

Die Dosierung mit den langkanüligen Spritzen erwies sich als sehr schwierig. Dadurch sank über die Dauer des Versuches hinweg der Flüssigkeitspegel. Undichtigkeiten mögen das Übrige dazu beigetragen haben. Dies führte zum kurzzeitigen Versagen der Cotrell-Pumpe. Der Flüssigkeitspegel wurde durch Zugabe von circa \SI{4}{\milli\liter} Cyclohexan und zwei Tropfen Ethanol angehoben. 

%\vspace*{5mm}

%\subsection*{Beispielfehlerrechnung für den ersten Messwert des rauen Rohres:}
%%Tabelle START
%\vspace*{-2.5mm}
%\renewcommand{\arraystretch}{1.2}
%\begin{table}[h!]
%	\centering
%	\caption{Abweichungen und Messwerte für die Fehlerrechnung}
%	\label{tab:abweichungen_daten}
%	\makebox[\textwidth]{
%	\begin{tabulary}{1.1\textwidth}{C|CC}
%		\hline 
%		\textbf{Messgröße} &\textbf{Messwert \linebreak(1, raues Rohr)} &\textbf{Abweichung} \\ 
%		\hline 
%		Volumenstrom & \SI{958}{\liter\per\hour} &$\pm2,5\%+MW \approx \SI{6,65e-6}{\raiseto{3}\meter \per \second}$\\ 
%		Temperatur &\SI{26,5}{\celsius} &$\pm \SI{0.5}{\kelvin} $\\ 
%		Druckmessungen & \SI{0,06}{\bar} &$2* \pm \SI{2}{\milli \mws}\approx \SI{4079}{\pascal}$\\ 
%		Durchmesser	&	\SI{13,6}{\milli \meter}&$ \pm 0$\\
%		Länge	&	\SI{2,5}{\meter}&$\pm 0$ \\
%		\hline 
%	\end{tabulary}
%	}
%\end{table}
%\FloatBarrier
%\vspace*{-5mm}
%%Tabelle Ende
%\begin{flalign}
%	\Delta p_v 	&= \frac{1}{2}*\frac{l}{d}*\rho(T)*v^2\\[5pt]
%	\lambda 	&= \frac{2*\Delta p_v*d}{l*\rho(T)*v^2}\\[5pt]
%	\label{gl:lambda}
%				&= \frac{2*\Delta p_v*d}{l*\rho(T)*\left(\frac{\dot{V}}{A}\right)^2}
%\end{flalign}
%
%Im Weiteren ist die eigentliche Fehlerrechnung für den ersten Messwert, der Messreihe des rauen Rohres, von $\lambda$ über das totale Differential der Gleichung \ref{gl:lambda} aufgeführt. Wichtig ist dabei zu erwähnen, dass alle Variablen in SI-Einheiten einzusetzen sind bis auf die Temperatur, welche in $\left[\si{\celsius}\right]$ eingesetzt wird.
%
%\subsection*{Bildung der Differentiale:}
%\begin{flalign}
%\frac{\partial \lambda}{\partial \Delta p_v} &= \frac{2*d*A^2}{l*\rho(T)*\dot{V}^2} = \frac{d^5*\pi^2}{8*l*\rho(T)*\dot{V}^2}\\[2mm]
%				&= \frac{1250*d^5*\pi^2*\left[\si{\kelvin \raiseto{3} \meter}\right]}{l*(-2683*T+10038000*\left[\si{\kelvin}\right])*\dot{V}^2*\left[\si{\kg}\right]}
%\end{flalign}
%\begin{flalign}
%	\frac{\partial \lambda}{\partial \dot{V}}	&= -\frac{4*\Delta p_v*d*A^2}{l*\rho(T)*\dot{V}^3}=-\frac{\Delta p_v*d^5*\pi^2}{4*l*\rho(T)*\dot{V}^3}\\[2mm]
%					&=- \frac{2500*\Delta p_v*d^5*\pi^2*\left[\si{\kelvin \raiseto{3} \meter}\right]}{l*(-2683*T+10038000*\left[\si{\kelvin}\right])*\dot{V}^3*\left[\si{\kg}\right]}\\[6mm]
%	\frac{\partial \lambda}{\partial T}	&=  \frac{3353750*\Delta p_v*d^5*\pi^2*\left[\si{\kelvin \raiseto{3} \meter}\right]}{l*(-2683*T+10038000*\left[\si{\kelvin}\right])*\dot{V}^2*\left[\si{\kg}\right]}
%\end{flalign}
%
%%Tabelle START
%
%\vspace*{-2.5mm}
%\renewcommand{\arraystretch}{1.2}
%\begin{table}[h!]
%	\centering
%	\caption{Ergebnisse der einzelnen Differentiale für den Messwert 1 des rauen Rohres}
%	\label{tab:differentiale}
%	%\resizebox{10cm}{!}{
%	\begin{tabulary}{\textwidth}{L|CCC}
%		\hline
%		\textbf{Differenzial} & $\frac{\partial \lambda}{\partial \Delta p_v}$ & $\frac{\partial \lambda}{\partial \dot{V}}$ &$ \frac{\partial \lambda}{\partial T }$\\ 
%		\hline
%			&\SI{3,25e-6}{\meter \raiseto{2}\second\per\kg}&\SI{-146,69}{\second\per\raiseto{3}\meter}&\SI{5,25e-6}{\per\kelvin}\\
%		\hline
%	\end{tabulary}
%	%}
%\end{table}
%\FloatBarrier
%%Tabelle Ende
%
%
%\subsection*{Berechnung des absoluten Fehlers:}
%\begin{flalign}
%	\Delta \lambda	&=  \left|\frac{\partial \lambda}{\partial \Delta p_v}\right|*\Delta p + \left|\frac{\partial \lambda}{\partial \dot{V}}\right|*\Delta \dot{V} + \left|\frac{\partial \lambda}{\partial T}\right|*\Delta T\\
%					&= \left|\SI{3,25e-6}{\meter \raiseto{2}\second\per\kg}\right|*\SI{4079}{\pascal}+ \left|\SI{-146,69}{\second\per\raiseto{3}\meter}\right|\SI{6,65e-6}{\raiseto{3}\meter \per \second}\\
%					&\nonumber \quad+\left|\SI{5,25e-6}{\per\kelvin}\right|*\SI{0,5}{\kelvin}\\[2mm]
%					&= \underline{\SI{0.0142}{}}
%\end{flalign}
%
%\subsection*{Berechnung des relativen Fehlers:}
%\begin{flalign}
%	\frac{\Delta \lambda}{\lambda}	&= \frac{0,0142}{0,0182}\\
%									&\approx \underline{\underline{\SI{78}{\percent}}}
%\end{flalign}
%%Tabelle START
%	\vspace*{-10.5mm}
%	\renewcommand{\arraystretch}{1.2}
%	\begin{table}[h!]
%		\centering
%		\caption{Absolute und relative Fehler von $\lambda$}
%		\label{tab:fehler}
%		%\resizebox{12.6cm}{!}{
%		\makebox[\textwidth]{%
%		\begin{tabular}{c|c|c|c}
%			\textbf{Messpunkt}	& \textbf{Rohrleitungswiderstand} &\textbf{Absoluter Fehler} $\left[-\right]$& \textbf{Relativer Fehler} $\left[\si{\percent}\right]$\\
%			\hline
%			\multicolumn{4}{l}{raues Rohr} \\
%			\hline
%			1&0,018&0,0142&78\\
%			2&0,020&0,0070&34\\
%			3&0,019&0,0041&21\\
%			4&0,021&0,0037&18\\
%			5&0,020&0,0031&16\\
%			\hline
%			\multicolumn{4}{l}{glattes Rohr} \\
%			\hline
%			1&0,026&0,0035&22\\
%			2&0,026&0,0029&19\\
%			(3)&(0,020)&(0,0020)&(17)\\
%			4&0,025&0,0025&17\\
%			5&0,025&0,0022&15\\
%			\hline
%			\multicolumn{4}{l}{glattes, dickes Rohr} \\
%			\hline
%			1&0,031&0,211&68\\
%			2&0,029&0,0156&54\\
%			3&0,033&0,0126&38\\
%			4&0,031&0,0108&35\\
%			5&0,031&0,0089&29\\
%			\hline
%		\end{tabular}}
%	\end{table}
%	\FloatBarrier
%	\vspace*{-2.5mm}
%	%Tabelle Ende
