\section{Versuchsdurchführung}
\label{sec:durchfuerung}

Die Versuchsdurchführung begann mit dem Ablesen des Atmosphärendrucks an einem Quecksilber-Präzisionsbarometer. Der abgelesene Luftdruck wurde hernach unter Zuhilfenahme des Computerprogramms \textsc{Baro} korrigiert. Außerdem wurde der Luftdruck noch an der Druckanzeige des Manostaten abgelesen.
Das Ebulliometer war bereits mit Isopropanol befüllt. Es galt nun den Manostaten, den Kryostaten für die Kühlwasserbereitstellung, den Magnetrührer und den Computer für die Anzeige der Temperaturen zu starten. Schließlich wurde auch der elektrische Heizmantel durch einschalten des Stelltrafos in Betrieb genommen.
Grundprinzip der Messungen war die Ermittlung der Siedetemperatur bei verschiedenen Drücken über dem Flüssigkeitsspiegel. Die erste Messung erfolgte bei \SI{100}{\kilo\pascal}. Anschließend wurde der Druck in 11 Schritten bis auf einen absoluten Druck von \SI{10}{\kilo\pascal} abgesenkt und die entsprechenden Siedetemperaturen aufgenommen. Zum Einstellen eines neuen Druckes musste die Stromversorgung des Heizmantels unterbrochen und das Ventil zum Manostaten geschlossen werden. Nachdem am Manostat der neue, gewünschte Druck eingestellt war, konnte man das Ventil langsam und vorsichtig wieder öffnen. Bei jeder Druckerniedrigung war ein starkes aufkochen des Isopropanols zu beobachten. Ein zu schnelles öffnen des Ventils hätte einen Austritt flüssigen Isopropanols aus dem Kühler zur Folge gehabt, da der sich plötzlich vergrößernde Gasstrom selbigen mitgerissen hätte. Es galt stets die Einstellung des thermischen Gleichgewichtszustandes abzuwarten, bevor die Temperatur abgelesen werden konnte. Am Computerbildschirm war gut zu erkennen wann der Graph im live-Temperatur-Zeit-Diagramm einen horizontalen Verlauf erreichte. 
Die Anlage wurde unter Aufsicht des Betreuers heruntergefahren. Dazu wurde die Heizung abgestellt und der Druck ausgeglichen. Es ist dabei zu beachten, dass der Druck im Siederaum des Ebulliometers nur langsam an den Atmosphärendruck angeglichen werden darf.
\begin{figure}[h!]
\centering
\includegraphics[width=0.7\linewidth]{img/versuchsaufbau-ebull}
\caption{Skizze zum Versuchsaufbau}
\label{fig:versuchsaufbau-ebull}
\end{figure}
