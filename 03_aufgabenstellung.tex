\section{Einleitung und Versuchsziel}
\label{sec:aufgabenstellung}
%In der Aufgabenstellung wird (in eigenen Worten und ganzen Sätzen) formuliert, was das Ziel des 
%Versuches ist.  
%[Beachten Sie die eigentliche Aufgabenstellung in den Versuchsanleitungen sowie die Hinweise zur Auswertung!] 

In diesem Versuch wurden Untersuchungen zum Siedeverhalten von reinem Isopropanol mit einem Ebulliometer angestellt. Ziel des Versuches ist das Erstellen einer Dampfdruckkurve und der Vergleich der erhaltenen Daten mit Werten aus der Literatur.

Das Wissen um die Dampfdrücke verschiedener Stoffe wird dringend zur Auslegung thermischer Trennverfahren wie etwa der Destillation und der Rektifikation benötigt. Aber auch bei der Nutzung des Dampf-Kraft-Prozesses und dem Umgang mit flüchtigen Chemikalien sind Dampfdruck-Temperatur-Abhängigkeiten zu beachten.
Der Dampfdruck sagt aus, bei welchem Druck sich ein Gleichgewichtszusand zwischen Flüssigkeit und Gasphase einstellt. Der Dampfdruck einer Substanz kann nur durch die Temperatur beeinflusst werden. Steigt die Temperatur, so steht den Molekülen in der Flüssigkeit mehr Energie zur Verfügung um die Phasengrenze zu überwinden. Es wechseln daher mehr Teilchen in die Gasphase und üben dann einen höheren Druck auf einander und die Gefäßwandung aus. Die Siedetemperatur beschreibt den Zustand, wenn der Dampfdruck einer Flüssigkeit den Umgebungsdruck überschreitet. Während des Siedens einer Flüssigkeit bleibt deren Temperatur konstant. Die Verdampfungsenthalpie sorgt für eine Energieabfuhr, in dem die überschüssige thermische Energie zum Wechsel des Aggregatzustandes genutzt wird. 

Zur Beschreibung des Dampfdruckes wurden im Laufe der Zeit immer neue Gleichungen entwickelt.\\
Die einfachste ist die Clapeyron-Gleichung (\ref{gl:clapeyron}). In ihr wird der Zusammenhang des Sättigungsdampfdruckes $p^\circ$ mit der Temperatur $T$ und den stoffspezifischen Konstanten $A$ und $B$ ausgedrückt. Sie gilt für jedes Phasengleichgewicht eines reinen Stoffes.

Die Clausius-Clapeyron-Gleichung \eqref{gl:CCG} wird im weiteren Verlauf unter Anderem zur Berechnung der molaren Verdampfungsenthalpie genutzt. Sie kann, wie auch die August'sche Dampfdruckgleichung \eqref{august} aus der Clapeyron-Gleichung abgeleitet werden.

Als Modell zur Beschreibung des Sättigungsdampfdruckes in Abhängigkeit wird die Antoine-Gleichung \eqref{gl:anton} genutzt. Diese basiert ebenfalls auf der Clapeyron-Gleichung. Sie ist aber besser zur Beschreibung realer Systeme geeignet, da in ihr ein dritter Stoffparameter C einbezogen wird. Die Antoine-Parameter A,B und C sind für viele Systeme bereits tabelliert. Dabei ist stets auf den temperaturabhängigen Geltungsbereich zu achten.
\begin{equation}\label{gl:clapeyron}
ln(p^\circ)=A-\frac{B}{T}
\end{equation}
\begin{flalign}\label{gl:anton}
	\lg(p)=A-\frac{B}{C+\vartheta[\si{\degreeCelsius}]}
\end{flalign}

