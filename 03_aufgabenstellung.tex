\section{Einleitung und Versuchsziel}
\label{sec:aufgabenstellung}
Die Destillation und Rektifikation von Stoffgemischen ist ein essentielles Trennverfahren in der chemischen Industrie. Um Berechnungen dazu ausführen zu können muss das Dampf-Flüssig-Gleichgewicht bekannt sein. Eben diese Abhängigkeiten zwischen Temperatur, Druck und den Zusammensetzungen der flüssigen und der Dampfphase soll in im Laborversuch, anhand eines binären Gemisches aus Ethanol und Cyclohexan, untersucht werden. Aus den gewonnenen Daten werden das Siede-/Taudiagramm, das Gleichgewichtsdiagramm, die Partialdrücke und die Aktivitätskoeffizienten abgeleitet. 

\section{Theoretische Grundlagen}

\subsection{Der Azeotrope Punkt}
Ein Azeotrop ist ein Gemisch aus mindestens zwei Komponenten in einem Bestimmten Mischungsverhältnis. Die Besonderheit ist, dass sich dieses Gemsich wie ein Reinstoff verhält. Es lässt sich nicht ohne weiteres Destillieren um die Komponenten zu trennen.
\subsection{Refraktometrie}
\subsubsection*{Das Abbe-Refraktometer}
\subsubsection*{Die Na-D-Linie}

\subsection{Die Gibbs'sche Phasenregel}

\subsubsection*{Anzahl der Freiheitsgrade}
\subsubsection*{Variablen des Systems}


\subsection{Raoult-Dalton'sches-Gesetz}
\subsubsection*{für ideale Mischungen}
\subsubsection*{für reale Mischungen}

\subsection{Thermodynamische Konsistenz}
\subsubsection*{Grundlegende Gleichung}
\subsubsection*{Warum GE modelle Grundsätzlich konsistent}
